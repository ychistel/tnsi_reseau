\documentclass[11pt,a4paper]{article}

\usepackage{style2017}
\usepackage{hyperref}

\hypersetup{
    colorlinks =false,
    linkcolor=blue,
   linkbordercolor = 1 0 0
}
\newcounter{numexo}
\setcellgapes{1pt}

\begin{document}



\begin{NSI}
{Exercice}{Protocoles de routage RIP et OSPF}
\end{NSI}




\addtocounter{numexo}{1}
\subsection*{\Large Exercice \thenumexo}

On considère le réseau composé de 4 routeurs A, B , C et D reliés selon le graphe ci-dessous:

\begin{center}
\psset{unit=1cm}
\begin{pspicture}(10,6)
\cnodeput(0.5,3.5){A}{A}
\cnodeput(4.5,5.5){B}{B}
\cnodeput(4.5,1){C}{C}
\cnodeput(8,2.5){D}{D}
\ncline{A}{B}%\ncput*{\bf 2}
\ncline{A}{C}%\ncput*{\bf 4}
\ncline{B}{C}%\ncput*{\bf 4}
\ncline{C}{D}%\ncput*{\bf 6}
%\ncline{R3}{R5}\ncput*{\bf 5}
%\ncline{R2}{R5}\ncput*{\bf 2}
%\ncline{R4}{R5}\ncput*{\bf 1}
%\ncline{R4}{R6}\ncput*{\bf 3}
%\ncline{R5}{R6}\ncput*{\bf 5}
\end{pspicture}
\end{center} 

Le protocole RIP est utilisé au sein de ce réseau.
\begin{enumerate}
\item Dresser la table de routage de chaque routeur en indiquant la destination, la passerelle (routeur suivant) et la distance (nombres de sauts) le séparant d'un autre routeur.
\item On relie un nouveau routeur E au routeur B. Donner la table de routage du routeur E et modifier les tables de routage des autres routeurs selon le protocole RIP.
\item Pour éviter l'isolement du routeur D, on le relie au nouveau routeur E. Corriger les tables de routage de chaque routeur.
\item On effectue des travaux sur le lien entre les routeurs C et D le rendant inopérant. Quelle sera alors la table de routage du routeur A selon le protocole RIP ?
\end{enumerate}


\addtocounter{numexo}{1}
\subsection*{\Large Exercice \thenumexo}

On donne les tables de routage de 4 routeurs reliés dans un même réseau contenant 5 routeurs.\medskip

Table de routage : \textbf{Routeur A}\hspace{4.8cm}Table de routage : \textbf{Routeur B}\medskip

\begin{tabular}{*{3}{|C{2.4cm}}|}\hline
Destination & Passerelle & Distance\\\hline
B & B & 1\\
C & C & 1\\
D & C & 2\\
E & B & 2\\\hline
\end{tabular}\hfill
\begin{tabular}{*{3}{|C{2.4cm}}|}\hline
Destination & Passerelle & Distance\\\hline
A & A & 1\\
C & A & 2\\
D & E & 2\\
E & E & 1\\\hline
\end{tabular}
\medskip

Table de routage : \textbf{Routeur C}\hspace{4.8cm}Table de routage : \textbf{Routeur D}\medskip

\begin{tabular}{*{3}{|C{2.4cm}}|}\hline
Destination & Passerelle & Distance\\\hline
A & A & 1\\
B & A & 2\\
D & D & 1\\
E & D & 2\\\hline
\end{tabular}\hfill
\begin{tabular}{*{3}{|C{2.4cm}}|}\hline
Destination & Passerelle & Distance\\\hline
A & C & 2\\
B & E & 2\\
C & C & 1\\
E & E & 1\\\hline
\end{tabular}
\medskip
%\newpage
%Table de routage : \textbf{Routeur }E\medskip
%
%\begin{tabular}{*{3}{|C{2.4cm}}|}\hline
%Destination & Passerelle & Distance\\\hline
%A & & 1\\
%B & & 1\\
%C & C & 2\\
%D & B & 2\\\hline
%\end{tabular}

\begin{enumerate}
\item Le routeur A doit transmettre un message au routeur E en effectuant un minimum de sauts. Quel est le trajet parcouru ?
\item Donner une table de routage pour le routeur E.
\item Représenter par un graphe le réseau entre ces 5 routeurs.
\end{enumerate}


\addtocounter{numexo}{1}
\subsection*{\Large Exercice \thenumexo}
Le protocole OSPF est utilisé dans un réseau.
\begin{enumerate}
\item Exprimer en bits par seconde, puis en Gb/s un débit de 50 Mb/s.
\item La liaison entre deux routeurs A et B a un débit de 100 Mbits/s. Calculer le coût de la liaison.
\item Le coût de la liaison entre deux routeurs A et C vaut 2. Calculer le débit entre ces deux routeurs.
\end{enumerate}


\addtocounter{numexo}{1}
\subsection*{\Large Exercice \thenumexo}
On donne le graphe d'un réseau de 7 routeurs. Le protocole OSPF est utilisé entre ces routeurs pour communiquer.

\begin{center}
\psset{unit=1.5cm,nodesep=3pt,nrot=:U}
\begin{pspicture}(10,6)
\cnodeput(0.5,1.5){A}{A}
\cnodeput(2,5.5){B}{B}
\cnodeput(3.5,1.5){C}{C}
\cnodeput(5,5.5){D}{D}
\cnodeput(6.5,1.5){E}{E}
\cnodeput(8,5.5){F}{F}
\cnodeput(9.5,1.5){G}{G}
\ncline{A}{B}\naput*{\bf 1 Gb/s}
\ncline{A}{C}\nbput*{\bf 100 Mb/s}
\ncline{B}{C}\naput*{\bf 1 Gb/s}
\ncline{B}{D}\naput*{\bf 100 Mb/s}
\ncline{B}{E}\naput*{\bf 100 Mb/s}
\ncline{E}{F}\nbput[npos=0.3]{\bf 1 Gb/s}
\ncline{C}{E}\nbput*{\bf 100 Mb/s}
\ncline{D}{F}\naput*{\bf 100 Mb/s}
\ncline{D}{G}\naput[npos=0.3]{\bf 100 Mb/s}
\ncline{E}{G}\nbput*{\bf 100 Mb/s}
\ncline{F}{G}\naput*{\bf 1 Gb/s}
\end{pspicture}
\end{center} 

\begin{enumerate}
\item Reproduire le graphe en remplaçant chaque débit par son coût.
\item Déterminer le trajet qui a la plus faible coût entre les routeurs A et G. Justifier la réponse.
\end{enumerate}

%\begin{center}
%\psset{unit=1.5cm,nodesep=3pt,nrot=:U}
%\begin{pspicture}(10,6)
%\cnodeput(0.5,1.5){A}{A}
%\cnodeput(2,5.5){B}{B}
%\cnodeput(3.5,1.5){C}{C}
%\cnodeput(5,5.5){D}{D}
%\cnodeput(6.5,1.5){E}{E}
%\cnodeput(8,5.5){F}{F}
%\cnodeput(9.5,1.5){G}{G}
%\ncline{A}{B}\naput*{\bf 0,1}
%\ncline{A}{C}\nbput*{\bf 1}
%\ncline{B}{C}\naput*{\bf 0,1}
%\ncline{B}{D}\naput*{\bf 1}
%\ncline{B}{E}\naput*{\bf 1}
%\ncline{E}{F}\nbput[npos=0.35]{\bf 0,1}
%\ncline{C}{E}\nbput*{\bf 1}
%\ncline{D}{F}\naput*{\bf 1}
%\ncline{D}{G}\naput[npos=0.4]{\bf 1}
%\ncline{E}{G}\nbput*{\bf 1}
%\ncline{F}{G}\naput*{\bf 0,1}
%\end{pspicture}
%\end{center} 


\end{document}
